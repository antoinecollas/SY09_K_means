% !TeX spellcheck = fr_FR
\documentclass[a4paper,11pt,oneside,roman]{article}
    \usepackage[utf8]{inputenc}
    \usepackage[T1]{fontenc}
    \usepackage[top=2cm, left=2cm, right=2cm, bottom=2cm]{geometry}
    \usepackage[francais]{babel}
    \usepackage[hidelinks=true]{hyperref}
    \usepackage{listings}
    \usepackage{color}
    \usepackage{amsmath}
    \usepackage{graphicx}
    \usepackage{amssymb}
    \usepackage{natbib}
    \usepackage{float}
    \usepackage{hyperref}
    \usepackage{advdate}
    
    \floatplacement{figure}{H}
    
    \definecolor{dkgreen}{rgb}{0,0.6,0}
    \definecolor{gray}{rgb}{0.5,0.5,0.5}
    \definecolor{mauve}{rgb}{0.58,0,0.82}
    
    \linespread{1.3} %space between lines
    \setlength{\parskip}{1em}  %space between paragraphs
    
    \begin{document}
    
    \begin{titlepage}
    
        \newcommand{\HRule}{\rule{\linewidth}{0.5mm}} % Defines a new command for the horizontal lines, change thickness here
        
        \center % Center everything on the page
         
        %----------------------------------------------------------------------------------------
        %	HEADING SECTIONS
        %----------------------------------------------------------------------------------------
        
        \textsc{\LARGE Université de Technologie de Compiègne}\\[0.5cm] % Name of your university/college
        \textsc{\Large Génie informatique}\\[1.5cm] % Name of your university/college
        
        %----------------------------------------------------------------------------------------
        %	TITLE SECTION
        %----------------------------------------------------------------------------------------
        
        \HRule \\[0.4cm]
        { \huge \bfseries Rapport du projet de SY09}\\[0.4cm] % Title of your document
        \HRule \\[1.5cm]
         
        %----------------------------------------------------------------------------------------
        %	AUTHOR SECTION
        %----------------------------------------------------------------------------------------
        
        % If you don't want a supervisor, uncomment the two lines below and remove the section above
        \Large \emph{Authors:}\\
        Crauser \textsc{Julien} et Antoine \textsc{Collas}\\[3cm] % Your name
        
        %----------------------------------------------------------------------------------------
        %	DATE SECTION
        %----------------------------------------------------------------------------------------
        
        {\large \AdvanceDate[-4]\today}\\[4cm] % Date, change the \today to a set date if you want to be precise
        
        %----------------------------------------------------------------------------------------
        %	LOGO SECTION
        %----------------------------------------------------------------------------------------
        
        \includegraphics[width=0.5\textwidth]{imgs/logo_UTC_SU.jpg}\\[1cm] % Include a department/university logo - this will require the graphicx package
        
        %----------------------------------------------------------------------------------------
    
        \vfill % Fill the rest of the page with whitespace
        
    \end{titlepage}
    
    % \tableofcontents
    
    \pagebreak
        
    \section{Cuisine}
    \subsection*{Question 1}
    Le jeu de données présent dans le fichier recettes-pays.data contient 51 variables pour 26 individus.
    Nous pouvons donc noter que nous disposons de plus de variables que d'individus.
    Parmis ces 51 variables, une seule est qualitative (les origines des recettes) et 50 sont quantitaves.
    Ces dernières prennent leurs valeurs entre $0$ et $1$ ($0$ et $0.82$ pour être plus précis).
    Dans ce jeux de données les recettes ont été agrégées par origine, il y a donc 26 origines (une par ligne de notre tableau individus-variables).
    De plus, le jeux de données ne présente aucunes valeurs manquantes.
    
    \subsection*{Question 2}
    Nous réalisons une ACP sur le jeu de données. Nous obtenons 26 axes principaux: autant que d'individus.
    En effet, comme il y a moins d'individus que de variables, il est suffisant de prendre 26 axes pour représenter tous les individus.

    Nous les pourcentages d'inerties expliquée visibles sur la figure \ref{fig_acp}. 
    Les 3 premiers axes représentent $72\%$ de la variance du nuage de points.
    \begin{figure}
        \centering
        \includegraphics[width=0.5\textwidth]{imgs/acp.png}
        \caption{Pourcentages d'inertie expliquée}
        \label{fig_acp}
    \end{figure}

    Comme les 3 premiers axes factoriels représentent $72\%$ de la variance totale, nous analysons le nuage de points seulement avec les 3 premiers plans factoriels.
    Nous obtenons les plans factoriels suivant:
    \begin{figure}
        \centering
        \includegraphics[width=0.5\textwidth]{imgs/acp_plan_1_2.png}
        \caption{Plan 1,2}
        \label{fig_acp_plan_1_2}
    \end{figure}
    \begin{figure}
        \centering
        \includegraphics[width=0.5\textwidth]{imgs/acp_plan_1_3.png}
        \caption{Plan 1,3}
        \label{fig_acp_plan_1_3}
    \end{figure}
    \begin{figure}
        \centering
        \includegraphics[width=0.5\textwidth]{imgs/acp_plan_2_3.png}
        \caption{Plan 2,3}
        \label{fig_acp_plan_2_3}
    \end{figure}
    \end{document}